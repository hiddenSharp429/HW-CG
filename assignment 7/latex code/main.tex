%%%%%%%%%%%%%%%%%%%%%%%%%%%%%%%%%%%%%%%%
% main.tex
% STU课程报告模板正文
% 作者:ZiXian-Zhu
% 该模板仓库地址GitHub:https://github.com/hiddenSharp429/STU_report_template
% 参考来源:中科院论文模板、上海交通大学课程论文模板
%%%%%%%%%%%%%%%%%%%%%%%%%%%%%%%%%%%%%%%%

\documentclass[12pt]{article}

\usepackage{STUReport}
\usepackage{xcolor}
\usepackage{tabulary}
\usepackage{tikz}
\usepackage{listings}
\usepackage{amsmath}
\usepackage{graphicx}
\usepackage{float}
\usepackage{booktabs}
\usetikzlibrary{shapes,arrows,positioning}

% 新增颜色定义
\definecolor{highlight}{RGB}{135,206,250} 
\definecolor{important}{RGB}{255,0,0}
\definecolor{note}{RGB}{0,128,0}
\definecolor{codegreen}{rgb}{0,0.6,0}
\definecolor{codegray}{rgb}{0.5,0.5,0.5}
\definecolor{codepurple}{rgb}{0.58,0,0.82}
\definecolor{backcolour}{rgb}{0.95,0.95,0.92}

% 定义C#语言
\lstdefinelanguage{csharp}{
  language=[Sharp]C,
  morekeywords={var,string,get,set,class,namespace,new,using,this,public,private,protected,virtual,override},
  morecomment=[l]{//},
  morecomment=[s]{/*}{*/},
  morestring=[b]",
}

% 定义JavaScript语言
\lstdefinelanguage{js}{
  keywords={break,case,catch,continue,debugger,default,delete,do,else,false,finally,for,function,if,in,instanceof,new,null,return,switch,this,throw,true,try,typeof,var,void,while,with},
  morecomment=[l]{//},
  morecomment=[s]{/*}{*/},
  morestring=[b]',
  morestring=[b]",
  sensitive=true
}

% 代码样式设置
\lstdefinestyle{mystyle}{
    backgroundcolor=\color{backcolour},   
    commentstyle=\color{codegreen},
    keywordstyle=\color{magenta},
    numberstyle=\tiny\color{codegray},
    stringstyle=\color{codepurple},
    basicstyle=\ttfamily\footnotesize,
    breakatwhitespace=false,         
    breaklines=true,                 
    captionpos=b,                    
    keepspaces=true,                 
    numbers=left,                    
    numbersep=5pt,                  
    showspaces=false,                
    showstringspaces=false,
    showtabs=false,                  
    tabsize=2,
    frame=single
}

% 设置默认代码样式
\lstset{style=mystyle}

% 更改图表标题语言
\renewcommand{\figurename}{Figure}
\renewcommand{\tablename}{Table}

\begin{document}

%%%%%%%%%%%%%%%%%%%%%%%%%%%%%%%%%%%%%%%%
% cover.tex
% STU课程报告封面
% 作者:ZiXian-Zhu
% 该模板仓库地址GitHub:https://github.com/hiddenSharp429/STU_report_template
% 参考来源:中科院论文模板、上海交通大学课程论文模板
%%%%%%%%%%%%%%%%%%%%%%%%%%%%%%%%%%%%%%%%

\title{CG Assignment 1}
\author{\textup{ZiXian-Zhu}}


\begin{titlepage}
    \newcommand{\HRule}{\rule{\linewidth}{0.5mm}}
    \pagecolor{gray!10} % 更改页面背景颜色
    \includegraphics[width=8cm]{figures/stu_logo_3-removebg-preview.png}\\[1cm] 
    \center 
    \quad\\[1.5cm]
    \textsl{\Large ShanTou University}\\[0.5cm] 
    \textsl{\large School of Mathematics and Computer Science}\\[0.5cm] 
    \makeatletter
    \HRule \\[0.4cm]
    { \huge \bfseries \@title}\\[0.4cm] 
    \HRule \\[1.5cm]
    \begin{minipage}{0.4\textwidth}
        \begin{flushleft} \large
            \emph{Author:}\\
            \@author 
        \end{flushleft}
    \end{minipage}
    ~
    \begin{minipage}{0.4\textwidth}
        \begin{flushright} \large
            \emph{Supervisor:} \\
            \textup{Prof Liao}
        \end{flushright}
    \end{minipage}\\[3cm]
    \makeatother
    {\large An Assignment submitted for the STU:}\\[0.5cm]
    {\large \emph{202420251001081 计算机图形学}}\\[0.5cm]
    {\large \today}\\[2cm] 
    \vfill 
\end{titlepage}




% 新增格式说明
\section*{Format Instructions}
this article uses the following format to emphasize different types of information:
\begin{itemize}
    \item \textcolor{highlight}{\textbf{blue bold}}: important concepts or definitions
    \item \textcolor{important}{\textbf{red bold}}: very important information
    \item \textcolor{note}{\textit{green italic}}: additional explanation or comment
    \item \underline{underline}: keywords or terms
\end{itemize}

%%%%%%%%% BODY TEXT %%%%%%%%%%%%%%%%%%%%%%%%%%%%%%%%%%%%%%%%

\section{Theoretical and Practical Analysis of 2D and 3D Transformations}\label{sec:transformations}

\subsection{Mathematical Foundation}

\textcolor{highlight}{\textbf{Transformation matrices}} are the fundamental building blocks for both 2D and 3D transformations. Let's examine their mathematical representations:

\begin{table}[h]
\centering
\begin{tabular}{ccc}
\toprule
\textbf{Operation} & \textbf{2D Matrix} & \textbf{3D Matrix} \\
\midrule
Translation & 
$\begin{bmatrix} 
1 & 0 & t_x \\
0 & 1 & t_y \\
0 & 0 & 1
\end{bmatrix}$ & 
$\begin{bmatrix}
1 & 0 & 0 & t_x \\
0 & 1 & 0 & t_y \\
0 & 0 & 1 & t_z \\
0 & 0 & 0 & 1
\end{bmatrix}$ \\
\midrule
Scaling & 
$\begin{bmatrix}
s_x & 0 & 0 \\
0 & s_y & 0 \\
0 & 0 & 1
\end{bmatrix}$ &
$\begin{bmatrix}
s_x & 0 & 0 & 0 \\
0 & s_y & 0 & 0 \\
0 & 0 & s_z & 0 \\
0 & 0 & 0 & 1
\end{bmatrix}$ \\
\midrule
Rotation & 
$\begin{bmatrix}
\cos\theta & -\sin\theta & 0 \\
\sin\theta & \cos\theta & 0 \\
0 & 0 & 1
\end{bmatrix}$ &
\textit{See rotation matrices below} \\
\bottomrule
\end{tabular}
\caption{Basic Transformation Matrices}
\label{tab:transformation-matrices}
\end{table}

\subsection{3D Rotation Matrices}
For 3D rotations, we have three fundamental rotation matrices around each axis:

\begin{enumerate}
    \item \textcolor{highlight}{\textbf{X-axis Rotation}}:
    \[
    R_x(\theta) = \begin{bmatrix}
    1 & 0 & 0 & 0 \\
    0 & \cos\theta & -\sin\theta & 0 \\
    0 & \sin\theta & \cos\theta & 0 \\
    0 & 0 & 0 & 1
    \end{bmatrix}
    \]

    \item \textcolor{highlight}{\textbf{Y-axis Rotation}}:
    \[
    R_y(\theta) = \begin{bmatrix}
    \cos\theta & 0 & \sin\theta & 0 \\
    0 & 1 & 0 & 0 \\
    -\sin\theta & 0 & \cos\theta & 0 \\
    0 & 0 & 0 & 1
    \end{bmatrix}
    \]

    \item \textcolor{highlight}{\textbf{Z-axis Rotation}}:
    \[
    R_z(\theta) = \begin{bmatrix}
    \cos\theta & -\sin\theta & 0 & 0 \\
    \sin\theta & \cos\theta & 0 & 0 \\
    0 & 0 & 1 & 0 \\
    0 & 0 & 0 & 1
    \end{bmatrix}
    \]
\end{enumerate}

\subsection{Key Theoretical Differences}

\begin{table}[h]
\centering
\begin{tabulary}{\textwidth}{LL}
\toprule
\textbf{2D Transformations} & \textbf{3D Transformations} \\
\midrule
Uses 3×3 homogeneous matrices & Uses 4×4 homogeneous matrices \\
Single rotation angle (around Z-axis) & Three rotation angles (Euler angles) or quaternions \\
Simpler perspective transformations & Complex perspective projections \\
No depth considerations & Requires Z-buffer for depth handling \\
\textcolor{note}{\textit{Linear computational complexity}} & \textcolor{note}{\textit{Higher computational complexity}} \\
\bottomrule
\end{tabulary}
\caption{Theoretical Comparison of 2D and 3D Transformations}
\label{tab:theoretical-comparison}
\end{table}

\subsection{Advanced Concepts}

\subsubsection{Homogeneous Coordinates}
\textcolor{important}{\textbf{Homogeneous coordinates}} are essential for both 2D and 3D transformations:
\begin{itemize}
    \item 2D point: $(x, y, w)$ where actual coordinates are $(x/w, y/w)$
    \item 3D point: $(x, y, z, w)$ where actual coordinates are $(x/w, y/w, z/w)$
    \item \textcolor{note}{\textit{Enables representation of infinite points and perspective transformations}}
\end{itemize}

\subsubsection{Quaternions in 3D Rotation}
\textcolor{important}{\textbf{Quaternions}} offer several advantages over Euler angles:
\begin{itemize}
    \item Avoid gimbal lock
    \item Smoother interpolation
    \item More compact representation
    \item Quaternion: $q = w + xi + yj + zk$ where $i^2 = j^2 = k^2 = ijk = -1$
\end{itemize}

\subsection{Practical Implementation Considerations}

\begin{table}[h]
\centering
\begin{tabulary}{\textwidth}{LLL}
\toprule
\textbf{Aspect} & \textbf{2D Implementation} & \textbf{3D Implementation} \\
\midrule
Memory Usage & 9 floating-point numbers & 16 floating-point numbers \\
Matrix Chain & Simple concatenation & Complex multiplication order \\
Performance & Fast, CPU-efficient & Often requires GPU acceleration \\
Precision & Less affected by floating-point errors & More susceptible to numerical errors \\
\bottomrule
\end{tabulary}
\caption{Implementation Considerations}
\label{tab:implementation-comparison}
\end{table}

\subsection{Common Applications and Use Cases}

\begin{enumerate}
    \item \textcolor{highlight}{\textbf{2D Applications}}:
        \begin{itemize}
            \item User interface elements
            \item Document layout
            \item 2D game sprites
            \item Vector graphics
        \end{itemize}
    
    \item \textcolor{highlight}{\textbf{3D Applications}}:
        \begin{itemize}
            \item Virtual reality
            \item 3D modeling and animation
            \item Scientific visualization
            \item Computer-aided design
        \end{itemize}
\end{enumerate}

\subsection{Performance Optimization Techniques}

\begin{enumerate}
    \item \textcolor{important}{\textbf{Matrix Optimization}}:
        \begin{itemize}
            \item Pre-computing common transformations
            \item Using specialized SIMD instructions
            \item Batch processing of transformations
        \end{itemize}
    
    \item \textcolor{important}{\textbf{Memory Management}}:
        \begin{itemize}
            \item Efficient matrix storage formats
            \item Cache-friendly data structures
            \item Memory alignment for SIMD operations
        \end{itemize}
\end{enumerate}

\subsection{Code Implementation Examples}

\subsubsection{OpenGL Matrix Operations}
\begin{figure}[H]
\begin{minipage}{\linewidth}
\begin{lstlisting}[language=C++, caption=OpenGL Matrix Transformations]
// 2D Translation
glm::mat3 transform2D = glm::translate(
    glm::mat3(1.0f), glm::vec2(x, y));

// 3D Translation with Rotation
glm::mat4 transform3D = glm::translate(
    glm::mat4(1.0f), glm::vec3(x, y, z));
transform3D = glm::rotate(
    transform3D, angle, glm::vec3(0,1,0));
\end{lstlisting}
\end{minipage}
\end{figure}

\subsubsection{Quaternion Rotation (Unity)}
\begin{figure}[H]
\begin{minipage}{\linewidth}
\begin{lstlisting}[language=csharp, caption=Unity Quaternion Operations]
// Creating a rotation
Quaternion rotation = Quaternion.Euler(x, y, z);
transform.rotation = rotation;

// Smooth rotation interpolation
transform.rotation = Quaternion.Slerp(
    startRotation, endRotation, time);
\end{lstlisting}
\end{minipage}
\end{figure}

\subsubsection{Python Implementation}
\begin{figure}[H]
\begin{minipage}{\linewidth}
\begin{lstlisting}[language=Python, caption=Python Matrix Operations]
import numpy as np

# 2D Rotation Matrix
def rotation_matrix_2d(theta):
    return np.array([
        [np.cos(theta), -np.sin(theta)],
        [np.sin(theta),  np.cos(theta)]
    ])

# 3D Rotation Matrix (around Y-axis)
def rotation_matrix_3d_y(theta):
    return np.array([
        [ np.cos(theta), 0, np.sin(theta)],
        [            0, 1,            0],
        [-np.sin(theta), 0, np.cos(theta)]
    ])
\end{lstlisting}
\end{minipage}
\end{figure}

\section{Practical Implementation of 2D and 3D Transformations}\label{sec:transformations}

\subsection{Popular Graphics Libraries and Frameworks}
Let's examine how different software implementations handle transformations:

\begin{enumerate}
    \item \textcolor{highlight}{\textbf{OpenGL}}:
        \begin{itemize}
            \item 2D: \texttt{glTranslatef(x, y, 0.0f)} for translation
            \item 3D: \texttt{glm::translate(model, glm::vec3(x, y, z))}
            \item \textcolor{note}{\textit{Uses GLM (OpenGL Mathematics) for matrix operations}}
        \end{itemize}
    
    \item \textcolor{highlight}{\textbf{Three.js (WebGL Framework)}}:
        \begin{itemize}
            \item 2D: \texttt{mesh.position.set(x, y, 0)}
            \item 3D: \texttt{mesh.position.set(x, y, z)}
            \item Rotation: \texttt{mesh.rotation.set(pitch, yaw, roll)}
            \item \textcolor{note}{\textit{Provides intuitive JavaScript API for 3D graphics}}
        \end{itemize}
        
    \item \textcolor{highlight}{\textbf{Unity Game Engine}}:
        \begin{itemize}
            \item 2D: \texttt{transform.Translate(new Vector2(x, y))}
            \item 3D: \texttt{transform.Translate(new Vector3(x, y, z))}
            \item \textcolor{note}{\textit{Supports both 2D and 3D game development}}
        \end{itemize}
\end{enumerate}

\subsection{Real-world Applications}

\begin{enumerate}
    \item \textcolor{important}{\textbf{Blender (Open Source 3D Software)}}:
        \begin{itemize}
            \item Uses quaternions for rotation: \texttt{obj.rotation\_quaternion}
            \item Matrix transformation: \texttt{obj.matrix\_world}
            \item Python API: \texttt{bpy.ops.transform}
        \end{itemize}
    
    \item \textcolor{important}{\textbf{AutoCAD}}:
        \begin{itemize}
            \item 2D commands: \texttt{MOVE}, \texttt{ROTATE}, \texttt{SCALE}
            \item 3D commands: \texttt{3DROTATE}, \texttt{3DSCALE}
            \item \textcolor{note}{\textit{Uses World Coordinate System (WCS)}}
        \end{itemize}
        
    \item \textcolor{important}{\textbf{Processing}}:
        \begin{itemize}
            \item 2D: \texttt{translate(x, y)}, \texttt{rotate(angle)}
            \item 3D: \texttt{translate(x, y, z)}, \texttt{rotateX/Y/Z(angle)}
            \item \textcolor{note}{\textit{Popular for creative coding and visualization}}
        \end{itemize}
\end{enumerate}

\subsection{Implementation Differences}

\begin{enumerate}
    \item \textcolor{highlight}{\textbf{Matrix Operations}}:
        \begin{itemize}
            \item 2D: OpenCV's \texttt{cv2.getRotationMatrix2D()}
            \item 3D: Eigen library's \texttt{Affine3d}
        \end{itemize}
    
    \item \textcolor{highlight}{\textbf{Performance Considerations}}:
        \begin{itemize}
            \item 2D: DirectX's \texttt{D2D1::Matrix3x2F}
            \item 3D: NVIDIA CUDA for parallel matrix operations
        \end{itemize}
    
    \item \textcolor{highlight}{\textbf{Graphics APIs}}:
        \begin{itemize}
            \item Vulkan: Low-level control with explicit matrix math
            \item Metal: Apple's framework for optimized transformations
        \end{itemize}
\end{enumerate}

\subsection{Practical Examples}

\begin{enumerate}
    \item \textbf{2D Game Development (PyGame)}:
    \begin{figure}[H]
    \begin{minipage}{\linewidth}
    \begin{lstlisting}[language=Python, caption=PyGame 2D Game Development]
    # 2D Sprite rotation
    sprite.angle += 45  # Rotate 45 degrees
    sprite.scale = (2, 2)  # Double size
    \end{lstlisting}
    \end{minipage}
    \end{figure}

    \item \textbf{3D Animation (Three.js)}:
    \begin{figure}[H]
    \begin{minipage}{\linewidth}
    \begin{lstlisting}[language=js, caption=Three.js 3D Animation]
    // 3D Object manipulation
    object.rotation.x += 0.01;
    object.position.z += 5;
    object.scale.set(2, 2, 2);
    \end{lstlisting}
    \end{minipage}
    \end{figure}
\end{enumerate}

\end{document}
