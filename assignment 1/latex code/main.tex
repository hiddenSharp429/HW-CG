%%%%%%%%%%%%%%%%%%%%%%%%%%%%%%%%%%%%%%%%
% main.tex
% STU课程报告模板正文
% 作者:ZiXian-Zhu
% 该模板仓库地址GitHub:https://github.com/hiddenSharp429/STU_report_template
% 参考来源:中科院论文模板、上海交通大学课程论文模板
%%%%%%%%%%%%%%%%%%%%%%%%%%%%%%%%%%%%%%%%

\documentclass[12pt]{article}

\usepackage{STUReport}
\usepackage{xcolor}
\usepackage{tabulary}
\usepackage{tikz}
\usetikzlibrary{shapes,arrows,positioning}

% 新增颜色定义
\definecolor{highlight}{RGB}{135,206,250} 
\definecolor{important}{RGB}{255,0,0}
\definecolor{note}{RGB}{0,128,0}

% 更改图表标题语言
\renewcommand{\figurename}{Figure}
\renewcommand{\tablename}{Table}

\begin{document}

%%%%%%%%%%%%%%%%%%%%%%%%%%%%%%%%%%%%%%%%
% cover.tex
% STU课程报告封面
% 作者:ZiXian-Zhu
% 该模板仓库地址GitHub:https://github.com/hiddenSharp429/STU_report_template
% 参考来源:中科院论文模板、上海交通大学课程论文模板
%%%%%%%%%%%%%%%%%%%%%%%%%%%%%%%%%%%%%%%%

\title{CG Assignment 1}
\author{\textup{ZiXian-Zhu}}


\begin{titlepage}
    \newcommand{\HRule}{\rule{\linewidth}{0.5mm}}
    \pagecolor{gray!10} % 更改页面背景颜色
    \includegraphics[width=8cm]{figures/stu_logo_3-removebg-preview.png}\\[1cm] 
    \center 
    \quad\\[1.5cm]
    \textsl{\Large ShanTou University}\\[0.5cm] 
    \textsl{\large School of Mathematics and Computer Science}\\[0.5cm] 
    \makeatletter
    \HRule \\[0.4cm]
    { \huge \bfseries \@title}\\[0.4cm] 
    \HRule \\[1.5cm]
    \begin{minipage}{0.4\textwidth}
        \begin{flushleft} \large
            \emph{Author:}\\
            \@author 
        \end{flushleft}
    \end{minipage}
    ~
    \begin{minipage}{0.4\textwidth}
        \begin{flushright} \large
            \emph{Supervisor:} \\
            \textup{Prof Liao}
        \end{flushright}
    \end{minipage}\\[3cm]
    \makeatother
    {\large An Assignment submitted for the STU:}\\[0.5cm]
    {\large \emph{202420251001081 计算机图形学}}\\[0.5cm]
    {\large \today}\\[2cm] 
    \vfill 
\end{titlepage}




% 新增格式说明
\section*{Format Instructions}
this article uses the following format to emphasize different types of information:
\begin{itemize}
    \item \textcolor{highlight}{\textbf{blue bold}}: important concepts or definitions
    \item \textcolor{important}{\textbf{red bold}}: very important information
    \item \textcolor{note}{\textit{green italic}}: additional explanation or comment
    \item \underline{underline}: keywords or terms
\end{itemize}

%%%%%%%%% BODY TEXT %%%%%%%%%%%%%%%%%%%%%%%%%%%%%%%%%%%%%%%%

\section{What are raster graphics and vector graphics? What features they have, respectively?}\label{sec:1}

\subsection{Raster Graphics vs. Vector Graphics}

\textcolor{highlight}{\textbf{Raster graphics and vector graphics}} are two fundamental types of digital images, each with distinct characteristics:

\begin{table}[h]
\centering
\begin{tabulary}{\textwidth}{|L|L|}
\hline
\textbf{Raster Graphics} & \textbf{Vector Graphics} \\
\hline
Composed of a grid of \underline{pixels} & Based on \underline{mathematical formulas} \\
\textcolor{important}{\textbf{Resolution-dependent}} & \textcolor{important}{\textbf{Resolution-independent}} \\
Ideal for complex, photorealistic images & Ideal for logos, illustrations, and typography \\
Common formats: JPEG, PNG, GIF, TIFF & Common formats: SVG, AI, EPS \\
\hline
\end{tabulary}
\caption{Comparison of Raster and Vector Graphics}
\label{tab:raster-vs-vector}
\end{table}

\subsubsection{Raster Graphics Features}
\begin{itemize}
    \item Can represent subtle color gradations
    \item \textcolor{note}{\textit{Larger file sizes for high-resolution images}}
    \item Quality loss when scaling up
    \item Well-suited for photographs and complex digital paintings
    \item Editing affects pixels directly
\end{itemize}

\subsubsection{Vector Graphics Features}
\begin{itemize}
    \item \textcolor{highlight}{\textbf{Infinitely scalable without quality loss}}
    \item Smaller file sizes for simple graphics
    \item Easy to edit individual elements
    \item Perfect for sharp edges and solid colors
    \item \textcolor{note}{\textit{Can be easily converted to raster graphics, but not vice versa}}
\end{itemize}

% It would be beneficial to include a figure here comparing raster and vector graphics
\begin{figure}[h]
    \centering
    % Include an image comparing raster and vector graphics
    \caption{Comparison of Raster and Vector Graphics}
    \label{fig:raster-vs-vector}
\end{figure}

\section{Draw architecture of a simple raster-graphics system and describe how it works.}\label{sec:2}
A simple raster graphics system typically consists of the following components and works as described below:

\begin{figure}[h]
\centering
\begin{tikzpicture}[node distance=2cm, auto, 
    block/.style={rectangle, draw, text width=2cm, text centered, minimum height=1cm},
    line/.style={draw, -latex'}]

    \node [block] (input) {Input Devices (Mouse, Keyboard)};
    \node [block, right=of input] (cpu) {CPU};
    \node [block, right=of cpu] (gpu) {GPU};
    \node [block, below=of cpu] (buffer) {Frame Buffer};
    \node [block, below=of buffer] (controller) {Display Controller};
    \node [block, below=of controller] (output) {Output Device (Monitor, Printer)};

    \path [line] (input) -- (cpu);
    \path [line] (cpu) -- (gpu);
    \path [line] (cpu) -- (buffer);
    \path [line] (gpu) -- (buffer);
    \path [line] (buffer) -- (controller);
    \path [line] (controller) -- (output);
\end{tikzpicture}
\caption{Architecture of a simple raster-graphics system}
\label{fig:raster-system}
\end{figure}

\textcolor{highlight}{\textbf{A simple raster graphics system}} typically consists of the following components:

\begin{table}[h]
\centering
\begin{tabulary}{\textwidth}{|L|L|}
\hline
\textbf{Component} & \textbf{Description} \\
\hline
Input Devices & Mouse, keyboard, graphics tablet \\
CPU & Central Processing Unit \\
GPU & Graphics Processing Unit \\
Frame Buffer & Temporary storage for rendered image \\
Display Controller & Manages output to display device \\
Output Device & Monitor, printer \\
\hline
\end{tabulary}
\caption{Components of a Raster Graphics System}
\label{tab:raster-system-components}
\end{table}

\textcolor{important}{\textbf{The system works as follows:}}
\begin{enumerate}
    \item User input is received through input devices
    \item The CPU processes the input and sends instructions to the GPU
    \item The GPU renders the image and stores it in the frame buffer
    \item The display controller reads from the frame buffer and sends signals to the output device
    \item The output device displays the rendered image
\end{enumerate}

% A diagram of this architecture would be very helpful here

\section{As we know, there are many kinds of graphics software in use in practice. Please list at least 4 kinds of graphics software for different application purpose.}\label{sec:3}

Graphics software plays a crucial role in various industries. Here are five major categories of graphics software, each serving different purposes:

\subsection{Raster Graphics Editors}
\begin{itemize}
    \item \textbf{Purpose:} \textcolor{highlight}{\textbf{Photo editing, digital painting, image manipulation, texture creation}}
    \item \textbf{Key Features:} Layer-based editing, filter libraries, brush customization, color correction
    \item \textbf{Examples:} Adobe Photoshop, GIMP, Corel Painter, Affinity Photo
\end{itemize}

\subsection{Vector Graphics Editors}
\begin{itemize}
    \item \textbf{Purpose:} \textcolor{highlight}{\textbf{Logo design, illustrations, typography, scalable graphics}}
    \item \textbf{Key Features:} Path editing, shape tools, advanced typography, SVG support
    \item \textbf{Examples:} Adobe Illustrator, Inkscape, CorelDRAW, Affinity Designer
\end{itemize}

\subsection{3D Modeling and Animation Software}
\begin{itemize}
    \item \textbf{Purpose:} \textcolor{highlight}{\textbf{3D modeling, animation, visual effects, game asset creation}}
    \item \textbf{Key Features:} Polygon modeling, rigging, texturing, particle systems, rendering engines
    \item \textbf{Examples:} Autodesk Maya, Blender, Cinema 4D, 3ds Max
\end{itemize}

\subsection{Computer-Aided Design (CAD) Software}
\begin{itemize}
    \item \textbf{Purpose:} \textcolor{highlight}{\textbf{Technical drawings, architectural design, product design}}
    \item \textbf{Key Features:} Precise measurements, 3D solid modeling, parametric design, simulation tools
    \item \textbf{Examples:} AutoCAD, SolidWorks, Fusion 360, Rhino
\end{itemize}

\subsection{Desktop Publishing Software}
\begin{itemize}
    \item \textbf{Purpose:} \textcolor{highlight}{\textbf{Page layout, document design, print media creation}}
    \item \textbf{Key Features:} Master pages, text flow, style sheets, typography controls, print preparation
    \item \textbf{Examples:} Adobe InDesign, QuarkXPress, Scribus, Affinity Publisher
\end{itemize}

\end{document}
